\documentclass{chi2011}
\usepackage{times}
% \usepackage{url}
\usepackage{graphics}
\usepackage{color}
\usepackage[pdftex]{hyperref}
\usepackage[backend=bibtex,style=verbose-trad2]{biblatex}
% \usepackage{apacite}

\hypersetup{%
    pdftitle={CMPT 481 Low-Fidelity Report},
pdfauthor={Birkenstocks \& Socks: Peggy Anderson, Chris Penner, Jonathan Baxter},
pdfkeywords={your keywords},
bookmarksnumbered,
pdfstartview={FitH},
colorlinks,
citecolor=black,
filecolor=black,
linkcolor=black,
urlcolor=black,
breaklinks=true,
}
\newcommand{\comment}[1]{}
\definecolor{Orange}{rgb}{1,0.5,0}
\newcommand{\todo}[1]{\textsf{\textbf{\textcolor{Orange}{[[#1]]}}}}

\pagenumbering{arabic}  % Arabic page numbers for submission.  Remove this line to eliminate page numbers for the camera ready copy

\bibliography{proposal}
\begin{document}
% To make various LaTeX processors do the right thing with page size.
\special{papersize=8.5in,11in}
\setlength{\paperheight}{11in}
\setlength{\paperwidth}{8.5in}
\setlength{\pdfpageheight}{\paperheight}
\setlength{\pdfpagewidth}{\paperwidth}

% Use this command to override the default ACM copyright statement
% (e.g. for preprints). Remove for camera ready copy.
% \toappear{Submitted for review to CHI 2011.}

\title{CMPT481 Low-Fidelity Report}
\numberofauthors{3}
\author{
\alignauthor Peggy Anderson\\
    \email{peggy.anderson@usask.ca}
    \alignauthor Chris Penner\\
    \email{clp848@mail.usask.ca}
    \alignauthor Jonathan Baxter\\
    \email{jab231@mail.usask.ca}
}


\maketitle

\section{Functionality}

Our app is designed to help users keep track of where they are spending their money and how they can better allocate
their resources. It does this by providing two functions; the ability to enter expenses and the ability to view past
spending. Our goal was to simplify the use of these functions as much as possible to encourage their use.

When entering an expense the user need only enter the category of the expense and its cost, the date of the expense is
inferred unless otherwise specified. To observe their past expenses the user may view charts representing their
spending in each category over specific time ranges (month, week, etc.).


\section{Prototype}
    \subsection{Overview}

    Our Prototype shows the simple interface that we designed to get users to
    input their expenses more often. We have focused on a responsive layout
    and so have designed both mobile and desktop prototype interfaces. Both
    of the interfaces have two different pages. The first is a Settings page
    which allows the user to name their categories and set their budget. The
    main page of our application shows what the user has spent on different
    categories using colour coded graphs where each category has its own
    colour. When a user wishes to add a new expense they click the plus sign in
    the top left corner and a menu will pop out. In this menu they will pick
    a category and enter the expense, with the cost rounded to the nearest
    dollar. This interface allows users to easily enter and track their
    spending.

    \subsection{Reflections}

    Creating the Low-fidelity prototype went well for our group. Before we met
    to start working on our final prototype we each came up with different
    ideas and layouts for the final product. When we met we took our favourite
    ideas from each of the sketches and put them together while making slight
    alterations. By the end we had both expanded each other's ideas. We looked at
    approximately 10 ideas before we settled on our final design.
    We spent 2 hours working on our design combining our ideas.

\section{Usability Inspection}
    \subsection{Inspection Method}
    We decided to use the Cognitive Walkthrough to evaluate our prototype. We
	chose the Cognitive Walkthrough because of the limited number of tasks in
    our application. There are only three tasks that can be performed in our
    application and the Cognitive Walkthrough allows us to focus on each of these
	tasks and scrutinize the steps that a user would take to complete these
    tasks. This process helped us find some bugs in the interface that may not
    have been noticed otherwise.

    \subsection{Task Examples}
    \begin{itemize}
        \item Enter a \$200 weekly budget into settings screen, setting names for categories and assigning any limits
            they wish to have.
        \item As a user, you have just purchased a pack of gum, enter it as an expense in the application.
        \item View expenses in the food category over previous month.
    \end{itemize}

    \subsection{Method}

    The inspection was conducted by our three group members, each of us worked through one of the given tasks. We
    tried our best to approach the task as a new user. Another group members controlled the prototypes as the user
    explained their intent. This also helped us to determine disparities in how we expected the application to behave.
    The third group member took notes on issues or confusion that were encountered throughout the tasks.

    \subsection{Results} 
    
    As a result of our testing we found several areas which were either not
    as clear to the user in our prototypes as we had initially expected or
    simply did not work towards our goals for the application. In fact, as we
    proceeded through the 'Settings' task we realized that our understanding of
    the way that we would achieve our application's goals was different amongst
    group members, and that the interface we built in the prototype was missing
    a few of our desired use-cases. This sparked a re-design which we will be
    using in our medium-fidelity prototype.

    There were also several other small things that we noticed, such as inputs
    that may have looked like buttons, and some areas where the desired flow
    through the page was not clear (e.g. which button do I push to submit or to
    save?). We also noticed a few workflow concerns, for instance if the user
    wished to cancel changes to the settings of the application there was no
    'Cancel' button to do so.

    There were a few areas where we missing headings or labels which would help
    to clarify the meaning of controls or settings. For instance we needed to
    add some clarifying headings as per the scope of the budget we were setting
    (e.g. is the budget per week or per month?).

	When the user was interacting with the main pages these were easy to use.
	There was no confusion when adding an expense or when they were asked to
	read what an expense was. The Low-Fidelity prototype did very well to meet
	the Functionality that we originally set out for our application. 


\section{Redesign}

Overall we have found that the application is quite usable and our users were able to complete each of the tasks we
assigned. We were pleasantly surprised how few interactions were required to complete each tasks, this was especially
encouraging because we focused on reducing clutter by focusing on a few very specific features. We only allow a few
small user customizations (to reduce friction and improve experience) and the users found the level of
customization to be sufficient for their use cases. In the cases where the users became confused we were able to
identify very simple fixes to solve the problem (which are enumerated below).

    \begin{itemize}
        \item Ensure Text fields are distinct and do not look like buttons.
        \item Remove extraneous buttons so that the interface is less confusing and the call to action is clear.
            Specifically have only one Submit/Save button on Settings screen.
        \item Allow cancellation of actions to improve user experience.
        \item Add labels and/or tooltips to clarify inputs and actions, (e.g. Specifically clarify scope of the budget
            (weekly, monthly, etc.))
        \item Unify interface by making add expense button and settings button symmetrical.
    \end{itemize}
\end{document}
