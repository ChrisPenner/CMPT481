\documentclass{chi2011}
\usepackage{times}
% \usepackage{url}
\usepackage{graphics}
\usepackage{color}
\usepackage[pdftex]{hyperref}
\usepackage[backend=bibtex,style=verbose-trad2]{biblatex}
% \usepackage{apacite}

\hypersetup{%
    pdftitle={CMPT 481 Low-Fidelity Report},
pdfauthor={Birkenstocks \& Socks: Peggy Anderson, Chris Penner, Jonathan Baxter},
pdfkeywords={your keywords},
bookmarksnumbered,
pdfstartview={FitH},
colorlinks,
citecolor=black,
filecolor=black,
linkcolor=black,
urlcolor=black,
breaklinks=true,
}
\newcommand{\comment}[1]{}
\definecolor{Orange}{rgb}{1,0.5,0}
\newcommand{\todo}[1]{\textsf{\textbf{\textcolor{Orange}{[[#1]]}}}}

\pagenumbering{arabic}  % Arabic page numbers for submission.  Remove this line to eliminate page numbers for the camera ready copy

\bibliography{proposal}
\begin{document}
% To make various LaTeX processors do the right thing with page size.
\special{papersize=8.5in,11in}
\setlength{\paperheight}{11in}
\setlength{\paperwidth}{8.5in}
\setlength{\pdfpageheight}{\paperheight}
\setlength{\pdfpagewidth}{\paperwidth}

% Use this command to override the default ACM copyright statement
% (e.g. for preprints). Remove for camera ready copy.
% \toappear{Submitted for review to CHI 2011.}

\title{CMPT481 Low-Fidelity Report}
\numberofauthors{3}
\author{
\alignauthor Peggy Anderson\\
    \email{peggy.anderson@usask.ca}
    \alignauthor Chris Penner\\
    \email{clp848@mail.usask.ca}
    \alignauthor Jonathan Baxter\\
    \email{jab231@mail.usask.ca}
}


\maketitle

\section{Functionality}

Our app is designed to help users keep track of where they are spending their money and how they can better allocate
their resources. It does this by providing two functions; the ability to enter expenses and the ability to view past
spending. Our goal was to simplify the use of these functions as much as possible to encourage their use.

When entering an expense the user need only enter the category of the expense and its cost, the date of the expense is
inferred unless otherwise specified. To observe their past expenses the user may view charts representing their
spending in each category over specific time ranges (month, week, etc.).


\section{Prototype}
    \subsection{Overview}
	Our Prototype shows the simple interface that we designed to get users to
	input there expenses more often. Since we are using a web application for
	our software we designed both a mobile and a desktop interface. In both of
	the interfaces we have two different pages. The first is a Settings page,
	this allows the user to set the categories and their budget. The main page
	of our application shows the data on spending in different categories using
	colour coded graphs where each category has gets a colour. When a user
	wishes to add a new expense they click the plus sign in the top left corner
	and a menu will pop out. In this menu they will pick a category and enter
	the expense rounded to the nearest dollar. This interface will allow for
	users to easily enter and track there spending.
    \subsection{Reflections}
	Creating the Low-fidelity prototype went well for out group. Before we met
	to start working on our final prototype we each came up with different ideas
	and layouts for the final product. When we met we took our favourite ideas
	from each others sketches and put them together well making slight changes
	to the ideas here and there. We had both fully fleshed out idea's from each
	other as well as sketches of different ideas. In total probably 10 idea
	where thrown around before we settled on out final design. We spent about 2
	hours working on our design combining all the ideas that we had come up
	with.

\section{Usability Inspection}
    \subsection{Motivation}
    \subsection{Task Examples}
    \subsection{Method}
    \subsection{Results}

\section{Redesign}



\printbibliography

\end{document}
