\documentclass{chi2011}
\usepackage{times}
% \usepackage{url}
\usepackage{graphics}
\usepackage{color}
\usepackage[pdftex]{hyperref}
\hypersetup{%
pdftitle={Your Title},
pdfauthor={Your Authors},
pdfkeywords={your keywords},
bookmarksnumbered,
pdfstartview={FitH},
colorlinks,
citecolor=black,
filecolor=black,
linkcolor=black,
urlcolor=black,
breaklinks=true,
}
\newcommand{\comment}[1]{}
\definecolor{Orange}{rgb}{1,0.5,0}
\newcommand{\todo}[1]{\textsf{\textbf{\textcolor{Orange}{[[#1]]}}}}

\pagenumbering{arabic}  % Arabic page numbers for submission.  Remove this line to eliminate page numbers for the camera ready copy

\begin{document}
% To make various LaTeX processors do the right thing with page size.
\special{papersize=8.5in,11in}
\setlength{\paperheight}{11in}
\setlength{\paperwidth}{8.5in}
\setlength{\pdfpageheight}{\paperheight}
\setlength{\pdfpagewidth}{\paperwidth}

% Use this command to override the default ACM copyright statement
% (e.g. for preprints). Remove for camera ready copy.
% \toappear{Submitted for review to CHI 2011.}

\title{CMPT481 Medium-Fidelity Report}
\numberofauthors{3}
\author{
\alignauthor Peggy Anderson\\
    \email{peggy.anderson@usask.ca}
    \alignauthor Chris Penner\\
    \email{clp848@mail.usask.ca}
    \alignauthor Jonathan Baxter\\
    \email{jab231@mail.usask.ca}
}

\maketitle

\section{Evaluation Goals}

The primary goal of our project is for users to use our application to
continuously track their expenses. It is our belief that if the application
is made easy to use through an intuitive interface and navigation then the
user will also be able to input expenses quickly and efficiently. If the user
can quickly tell us how much they have spent in a specific category after
expenses have been entered then it confirms that the application is presenting
a clear visualization of the data. Unfortunately the continued use of the
application can not be tested within the scope of this class since that would
require multiple interviews with the test users to evaluate their usage and a
high-fidelity prototype.

	\subsection{Goals}
    We have ranked our goals on two metrics, importance is denoted by numbers, and the testability of each is denoted
    by letters. 
	\begin{enumerate}
	\item Continued use of the application (D)
	\item Navigation of the application is easy (B)
	\item Visualization of data is clear (C)
    \item Expenses can be entered quickly and accurately (A)
	\end{enumerate}


\section{Rationale for Type of Evaluation}

We will use interviews to evaluate our goals. Seeking to determine what they   
require to to make their budgeting experience with our app the most enjoyable. 

\section{Participant Pool}
We are using a mixture of family/friends for our test cases. 
\begin{itemize}
    \item User 1 - An individual who has limited experience with computers. 
	\item User 2 - A business student who has studied budgeting. 
	\item User 3 - An individual who has a degree in computer science and may offer some valuable criticisms. 
    \item User 4 - Older demographic who may have budgeted using other methods in the past.
\end{itemize}

\section{Overview of Interview Protocol}

We have set out a guideline of questions/topics we would like to cover with
each of the users but are not limited to the questions specified. We will
choose which questions to ask based on the user's responses. We will justify our
reasoning regarding which questions we choose to ask.

While the users are testing the application we will observe how quickly and
accurately they fufill the tasks. If they appear to be moving slowly or talking
to themselves during this process we will ask them to elaborate their thought
processes.

We will take notes during the user's demo and interview. The length of the
evaluation is anticipated to take less than 45 minutes. The quality of
the session will be evaluated based on the detail of answers provided.

	\subsection{Tasks we get the users to perform}
	\begin{itemize}
        \item  Add expenses to a few different categories
        \item  Navigate to data visualizations
        \item  Change the name of a category
        \item  Change the overall budget limit from the default
        \item  Change one category's weekly budget
        \item  Add expenses to a category to go over the budget's limit
	\end{itemize}

	\subsection{Interview Questions}
	Before the user performs the aforementioned tasks we will sit down with them and ask the following 
	questions:\\
    Do you currently or have you ever budget(ed)?
			\begin{itemize}
				\item (Yes - Probing Question) What method do you use?
				\item (No - Probing Question) Is there a specific reason why?
			\end{itemize}

    After performing the tasks provided on the Desktop interface:
	\begin{itemize}
		\item Was there anything thing that you found counter intuitive?
		\item If you could change/add things about the app, what would it be? (From desktop perspective)
	\end{itemize}

    After performing the tasks provided on the Mobile interface:
	\begin{itemize}
		\item Was there anything thing that you found counter intuitive?
		\item If you could change/add things about the app, what would it be? (From mobile perspective)
	\end{itemize}

	How did you find the mobile interface versus the desktop interface?
	\begin{itemize}
		\item (Probing Question) Which was easiest to use?
        \item (Probing Question) What made the data most visually clear for you?
	\end{itemize}
		
	Have you tried using other budgeting apps before, how did ours compare?
	\begin{itemize}
        \item (Yes - Probing Question) Which app allowed the fastest expense input?
        \item (Yes - Probing Question) Which app allowed most accurate expense input?
        \item (Yes - Probing Question) Are you more likely to continue using our app or a different one?
    \item (No to last question - Probing Question) Why?
	\end{itemize}
    (Probing Question) Would you be likely to continue using our app consistently? Why or why not?

	\section{Prototype Rational}

    To build our Medium-Fidelity Prototype we decided to use web technologies
    (html, css, javascript) with the React and Redux libraries. The choice was made primarily
    due to our group's comfort in working with these technologies, but also for accessability.
    By building the app using web technologies we were able to easily distribute our app for testing,
    and we can use the same responsive prototype on desktop and mobile devices.

    Using html/css allowed us to try different fonts, colours and themes
    and iterate quickly on functionality and styling so that we could get our
    application to look the way that we wanted it to. We also used a styling framework
    called Bulma to improve the responsiveness of our application for mobile users.

    Using React and Redux to build the prototype allows for a fully functioning prototype
    without tester intervention. This automates what would have otherwise been a "Wizard of Oz"
    approach. Using the javascript allows us to use animations to convey knowledge to the user
    and to draw attention to specific parts of the app.

    To simplify the implementation to fit the timeframe available we have not implemented any 
    data persistence, all expenses are lost once the application is closed.

\bibliographystyle{acm-sigchi}
\bibliography{sample}

\end{document}
