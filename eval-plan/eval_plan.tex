\documentclass{chi2011}
\usepackage{times}
% \usepackage{url}
\usepackage{graphics}
\usepackage{color}
\usepackage[pdftex]{hyperref}
\hypersetup{%
pdftitle={Your Title},
pdfauthor={Your Authors},
pdfkeywords={your keywords},
bookmarksnumbered,
pdfstartview={FitH},
colorlinks,
citecolor=black,
filecolor=black,
linkcolor=black,
urlcolor=black,
breaklinks=true,
}
\newcommand{\comment}[1]{}
\definecolor{Orange}{rgb}{1,0.5,0}
\newcommand{\todo}[1]{\textsf{\textbf{\textcolor{Orange}{[[#1]]}}}}

\pagenumbering{arabic}  % Arabic page numbers for submission.  Remove this line to eliminate page numbers for the camera ready copy

\begin{document}
% To make various LaTeX processors do the right thing with page size.
\special{papersize=8.5in,11in}
\setlength{\paperheight}{11in}
\setlength{\paperwidth}{8.5in}
\setlength{\pdfpageheight}{\paperheight}
\setlength{\pdfpagewidth}{\paperwidth}

% Use this command to override the default ACM copyright statement
% (e.g. for preprints). Remove for camera ready copy.
% \toappear{Submitted for review to CHI 2011.}

\title{CMPT481 Medium-Fidelity Report}
\numberofauthors{3}
\author{
\alignauthor Peggy Anderson\\
    \email{peggy.anderson@usask.ca}
    \alignauthor Chris Penner\\
    \email{clp848@mail.usask.ca}
    \alignauthor Jonathan Baxter\\
    \email{jab231@mail.usask.ca}
}

\maketitle

\section{Goals of Evaluation}

For our project the primary goal is to have a user want to continuously use the application
to track their expenses for the week/month/year. It is our belief that if the user is able 
to navigate the application intuitively and it is behaving as expected then the application 
is easy to use. If the application is easy to use then it can be said that the user will 
also be able to input expenses quickly and efficiently since they can intuitvely navigate 
to that screen. Once the user has inputted expenses if the user can quickly tell us how much
they have said they spent in a specific category accurately without hesitation then it can be
said that the application's visualization of data is clear and appealing. However, the 
continued use of the application is not testable within the scope of this class since that
would include multiple interviews with the test users to evaluate their usage and it would 
require a high-fidelity prototype. 
	\subsection{Goals}
	For our goals we have ranked importance by the use of numerical values, and we
	have ranked our ability to test them within the scope by the use of alphabetical
	values. 
	\begin{enumerate}
	\item Continued use of the application (D)
	\item Navigation of the application is easy (B)
	\item Visualization of data is clear and appealing (C)
	\item Ability to quickly and accurately input expenses (A)
	\end{enumerate}


\section{Rational for Type of Evaluation}
We will make use of interviews to evaluate our goals. By doing this we are creating that 
relationship with the user in attempts to determine what it is that they need to make their 
budgeting experience with our app the most enjoyable. 

\section{Participant Pool}
We are using a mix of family/friends for our test cases. 
\begin{itemize}
	\item Lauren - An individual who has limited experience with computers. 
	\item Mitchell - A business student who has studied budgeting. 
	\item Adam - An individual who has a degree in computer science and may offer some valuable criticisms. 
	\item One of Jonathan's Parents - Older demographic who may have budgeted using other methods in the past.
\end{itemize}

\section{Overview of Interview Protocol}
We have set out a guideline of questions/topics we would like to cover with each of the users but 
we are not limited to only asking what is below. We can choose to not ask a specific question if we
believe that the user has provided sufficient information for the question at any point during the
interview. In the event that we choose to not ask a question we will be required to support our 
reasoning whether it be due to the fact that the conversation flowed in other directions.

While the users are testing the application we will watch how quickly and accurately they are able to fufill the 
tasks. If they appear to be moving slowly or talking to themselves during this process we will ask them
to elaborate their thought process immediately.

We will analyze our data by taking notes during the user's demo and interview. 
The length of the evaluation is not anticipated to exceed 45 minutes, it will greatly depend on the
quality of answers from the user. Where the quality of the session will evaluated based on if their
answers provide lots of detail and information. If the users have much to say then it will be on 
the longer end, similarly if the user has difficulty navigating the system. 
	\subsection{Tasks we get the users to perform}
	\begin{itemize}
		\item adding different expenses to two or three different categories
		\item after adding expenses, (go to screen) for data visualization
		\item  changing the name of a category
		\item  changing the overall budget limit (it will originally be set to 200\$)
		\item  change one category limit
		\item  add more expenses to that category and get it to go over the budget limit
	\end{itemize}
	\subsection{Interview Questions}
	Before the user performs the aforementioned tasks we will sit down with them and ask the following 
	questions:\\
	Do you currently or have you ever budget(ed)?
			\begin{itemize}
				\item (Yes - Probing Question) What method do you use?
				\item (No - Probing Question) Is there a specific reason why?
			\end{itemize}
	First we will get the users to perform these tasks on a laptop/desktop browser. 
	\begin{itemize}
		\item Was there anything thing that you found counter intuitive?
		\item If you could change/add things about the app, what would it be? (From desktop perspective)
	\end{itemize}
	We will then proceed to get the users to perform these tasks on a mobile device of their 
	preference.
	\begin{itemize}
		\item Was there anything thing that you found counter intuitive?
		\item If you could change/add things about the app, what would it be? (From mobile perspective)
	\end{itemize}
	We will also ask:
	How did you find the mobile interface versus the desktop interface?
	\begin{itemize}
		\item (Probing Question) Which was easiest to use?
		\item (Probing Question) For you what made the data more visually clear for you?
	\end{itemize}
		
	(If they have budgeted before) Have you tried using other budgeting apps before, how did ours compare?
	\begin{itemize}
	\item (Yes - Probing Question) What app allowed you fast input?
	\item (Yes - Probing Question) What app allowed more accurate input?
	\item (Yes - Probing Question) Are you more likely to continue using ours or the other one?
	\item (No to last question - Probing Question) Why?
	\end{itemize}
	(Probing Question) After testing this app, would you be likely to continue using it consistently? 

	\section{Prototype Rational}

	To build our Medium-Fidelity Prototype we decided to use a mixture of
	html,css and the react/redux. Since our application is a web based
	application we decided that this would allow us to evaluate our prototype on
	both a desktop and a mobile platform. Using html/css allowed us to try
	different fonts, colours and themes so that we could get our application to
	look the way that we wanted it to. Using react we were able to adjust the
	look of the application depending on the platform it's being used on.

	Using react/redux javascript library let us automate the "Wizard of Oz"
	aspect of our application. Using the javascript we were able to animate the
	actions that a user would use and give the appearance that our application is
	fully functioning. We don't store any of the input data this is all done
	with local storage in the browser. The Medium-Fidelity prototype will let
	us easily evaluate our application with minimum interaction from the
	evaluator. 

\bibliographystyle{acm-sigchi}
\bibliography{sample}

\end{document}
