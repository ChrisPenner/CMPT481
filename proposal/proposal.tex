\documentclass{chi2011}
\usepackage{times}
% \usepackage{url}
\usepackage{graphics}
\usepackage{color}
\usepackage[pdftex]{hyperref}
\hypersetup{%
    pdftitle={CMPT 481 Project Proposal},
pdfauthor={Birkenstocks \& Socks: Peggy Anderson, Chris Penner, Jonathan Baxter},
pdfkeywords={your keywords},
bookmarksnumbered,
pdfstartview={FitH},
colorlinks,
citecolor=black,
filecolor=black,
linkcolor=black,
urlcolor=black,
breaklinks=true,
}
\newcommand{\comment}[1]{}
\definecolor{Orange}{rgb}{1,0.5,0}
\newcommand{\todo}[1]{\textsf{\textbf{\textcolor{Orange}{[[#1]]}}}}

\pagenumbering{arabic}  % Arabic page numbers for submission.  Remove this line to eliminate page numbers for the camera ready copy

\begin{document}
% To make various LaTeX processors do the right thing with page size.
\special{papersize=8.5in,11in}
\setlength{\paperheight}{11in}
\setlength{\paperwidth}{8.5in}
\setlength{\pdfpageheight}{\paperheight}
\setlength{\pdfpagewidth}{\paperwidth}

% Use this command to override the default ACM copyright statement
% (e.g. for preprints). Remove for camera ready copy.
% \toappear{Submitted for review to CHI 2011.}

\title{CMPT481 Project Proposal}
\numberofauthors{3}
\author{
\alignauthor Peggy Anderson\\
    \email{peggy.anderson@usask.ca}
    \alignauthor Chris Penner\\
    \email{clp848@mail.usask.ca}
    \alignauthor Jonathan Baxter\\
    \email{jab231@mail.usask.ca}
}


\maketitle

\begin{abstract}
    Polygons and stuff
\end{abstract}

\category{H.5.2}{Information Interfaces and Presentation}{Miscellaneous}[Optional sub-category]

\terms{
  See list of the limited ACM 16 terms in the instructions, see http://www.sheridanprinting.com/sigchi/generalterms.htm.
}

\section{Problem \& Motivation}

In 1980, the ratio for debt-to-income in Canada is 66\%; that ratio passed the 150\% figure in 2011. At a young age 
Canadians are taught how to count and spend money but we are not all taught how to budget effectively. Whether you're 
a business owner, student, or bringing home the bacon for your family; it's important to budget. Budgeting allows a 
person to determine if they will have enough money for what they need to do and what they would like to do. The reason
people typically tend to fall into debt is due to the fact that they're spending more than they expect to spend. If a 
person was to set aside a predetermined amount of money they'd spend a week/month and they were able to visually see what 
they're spending versus what they're saving if they don't spend the total amount then that person could potentially 
be more cautious of what they're spending recreationally.

Where people spend money every day it can be easy to forget to add smaller expenses or a large quantity of separate 
expenses. Reason being due to depending on their method of choice for budgeting, the input of their expenses may be 
difficult to execute in a timely fashion. With current applications that are purposed towards tracking 
weekly/monthly allocated spending funds on recreational purposes (non reoccuring bills/expenses,) they are not not 
very easy or engaging to use. Engaging in regards to the visualization used is not easy to distinctly see where the 
money is being spent, whether it be food/clothes/music/games/etc. Easy to use in regards to adding expenses, or navigating 
the application where there may be "too much going on". The portability of the application can also be an issue whether the 
usage is the same when using it on a desktop versus mobile device, between different mobile devices, and between different 
internet browsers. 



\begin{enumerate}
    \item Current Budgeting apps are not user friendly
    \item It's tough visualize how much we're spending on what
    \item It's tough to input expenses/costs, clunky interfaces and cost entry
    \item Difficulties discourage people from actually tracking their expenses
\end{enumerate}
 However, Current Applications for budgeting are not very easy to use or very
 engaging. With the current applications inputting spending and seeing how you
 are spending your money is not as easy or quick as it could be.  

(Include crappy interface photos here)

\begin{enumerate}
\item Want to display data clearly at a glance in a way that encourages action
\item Allow people to understand where they're using their money
\item Encourage saving money
\item Improve transparency in your budget
\item Reduce friction in inputting expenses so people actually do it.
\item Mobile friendly solution
\item Budgeting apps are only useful if people log ALL expenses, so it needs to be effortless.
\end{enumerate}

\section{Solution}

Our application seeks to address the aforementioned problems by implementing a
simple-to-use interface that streamlines the experience by omitting any
unnecessary frills and features in an attempt to focus the user on the areas
that are most important. Our application will feature a 'Quick Expense' screen
which allows users to enter an expense in its category in a matter of seconds.
We will experiment with the trade-off between accuracy and ease-of-use, for
example omitting the cents portion of cost inputs to simplify the interface.

Another component of ease-of-use is availability, which
we provide by implementing the system as a mobile friendly web application which
can be accessed from anywhere internet is available. This ensures that the user 
has access to the application at the time that they are completing transactions.

The primary action of viewing spending habits involves presenting data in a way
that is easy for the consumer to both view and act upon. The most important
metric of past spending is how the amount spent in each category relates to
their expected cost in that area. This application will allow views of expenses
over time filterable by category; the amount spent in the category will be
contrasted with the expected amount spent in that category over that time
interval as predicted by the user. This allows the user to determine areas where
there is a misalignment between their actions and intentions giving them an
actionable area of improvement.

\subsection{Steps to Solution}

Several steps are required to achieve our goals.  We will begin by prototyping
several user-interfaces which facilitate our primary actions. We may involve
external parties to determine which interfaces are the easiest to understand in
an attempt to make the interface as self-evident as possible. This stage should
result in a prototype for the 'Quick Expense' screen, one for the 'Expenses
Viewer' screen, and a notion for how to tie the two screens together.

At this point we may begin to consider our implementation, and will take time
to examine possible data models and system infrastructure. We should ensure
that our models are flexible enough to adapt to the inevitable changes that will
occur during the concrete implementation stage.

Once we have an understanding of how our application will be linked together we
may begin our implementation. We will start by implementing an MVP (minimal
viable product) which has the bare minimum functionality in each section so that
we can get a feel for how the application operates. We will then perform
progressive enhancement on each component until we are satisfied.

We will require a REST-ful API for any data that will be persisted. We will of
course need to implement the views and behaviour for the 'Quick Expense' and
'Expenses Viewer' screens. Since user experience is a top priority we will spend
significant time improving look and feel through styling and animations.

\section{Evaluation}
To evaluate how our web app works for the purpose of this class, we will compare it to a commercial
application. Users will be asked to do a series of tasks then we will use a
NASA tlx based questionnaire to evaluate the user experience on both
applications. 
\begin{enumerate}
    \item Qualitative comparisons of ease of use vs competitors
    \item Qualitative representation of how enjoyable navigating and using the app is.
	\item NASA TLX, give the users tasks and evaluate there experience with NASA tlx
\end{enumerate}

\bibliographystyle{acm-sigchi}
\bibliography{proposal}

\end{document}
