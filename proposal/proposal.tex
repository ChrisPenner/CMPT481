\documentclass{chi2011}
\usepackage{times}
% \usepackage{url}
\usepackage{graphics}
\usepackage{color}
\usepackage[pdftex]{hyperref}
\hypersetup{%
    pdftitle={CMPT 481 Project Proposal},
pdfauthor={Birkenstocks \& Socks: Peggy Anderson, Chris Penner, Jonathan Baxter},
pdfkeywords={your keywords},
bookmarksnumbered,
pdfstartview={FitH},
colorlinks,
citecolor=black,
filecolor=black,
linkcolor=black,
urlcolor=black,
breaklinks=true,
}
\newcommand{\comment}[1]{}
\definecolor{Orange}{rgb}{1,0.5,0}
\newcommand{\todo}[1]{\textsf{\textbf{\textcolor{Orange}{[[#1]]}}}}

\pagenumbering{arabic}  % Arabic page numbers for submission.  Remove this line to eliminate page numbers for the camera ready copy

\begin{document}
% To make various LaTeX processors do the right thing with page size.
\special{papersize=8.5in,11in}
\setlength{\paperheight}{11in}
\setlength{\paperwidth}{8.5in}
\setlength{\pdfpageheight}{\paperheight}
\setlength{\pdfpagewidth}{\paperwidth}

% Use this command to override the default ACM copyright statement
% (e.g. for preprints). Remove for camera ready copy.
% \toappear{Submitted for review to CHI 2011.}

\title{CMPT481 Project Proposal}
\numberofauthors{3}
\author{
\alignauthor Peggy Anderson\\
    \email{peggy.anderson@usask.ca}
    \alignauthor Chris Penner\\
    \email{clp848@mail.usask.ca}
    \alignauthor Jonathan Baxter\\
    \email{jab231@mail.usask.ca}
}


\maketitle

\section{Problem \& Motivation}

In 1980, the ratio for debt-to-income in Canada is 66\%; that ratio passed the 150\% figure in 2011. At a young age
Canadians are taught how to count and spend money but we are not all taught how to budget effectively. Whether you are
a business owner, student, or bringing home the bacon for your family; it is important to budget. Budgeting allows a
person to determine if they will have enough money for what they need to do and what they would like to do. The reason
people might fall into debt is the fact that they are spending more than they expected. If a person was to set aside a
predetermined amount of money that they wish to spend each week or month and could better visualize their spending
habits in relation to their saving. This will allow them to better prioritize their budget and find areas to increase
savings.

It can be easy to forget to add expenses as they occur throughout the day. Reasons for this may include that the method
of inputting their expenses is too time-consuming or difficult to use; or that their budgeting application is not
available on their mobile device. Many of the current applications that are purposed towards tracking weekly or monthly
recreational budgets (non re-occurring bills and expenses) are not easy to use or engaging for the user.

Most visualization techniques present in other applications focus solely on the amount of money spent in each category,
however this overlooks the primary purpose of budgeting and does not deliver the information of how the amount spent
relates to the users savings goals. Many applications sacrifice ease-of-use by adding clutter for additional (often
unused) features. When navigating many of these applications it can seem that there is "too much going on". 

Accessibility of the application can also be an issue, disparity in interface between the Desktop and mobile versions
can be a confusing experience. Some applications are not available on all devices.

\section{Solution}

Our application seeks to address the aforementioned problems by implementing a simple-to-use interface that streamlines
the experience by omitting any unnecessary frills and features in an attempt to focus the user on the areas that are
most important. Our application will feature a 'Quick Expense' screen which allows users to enter an expense in a
matter of seconds. We will experiment with the trade-off between accuracy and ease-of-use, for example omitting the
cents portion of cost inputs to simplify the interface.

In order to address the availability issue, we will implement the system as a mobile friendly web application which can
be accessed from anywhere internet is available on either desktop or mobile. This ensures that the user has access to
the application at the time that they are completing transactions.

The primary action of viewing spending habits involves presenting data in a way that is easy for the consumer to both
view and act upon. The most important metric of past spending is how the amount spent in each category relates to their
expected cost in that area. This application will allow views of expenses over time filterable by category; the amount
spent in the category will be contrasted with the expected amount spent in that category over that time interval as
predicted by the user. This allows the user to determine areas where there is a misalignment between their actions and
intentions and gives them an area to improve on.

\subsection{Steps to Solution}

Several steps are required to achieve our goals. We will begin by prototyping several user-interfaces which facilitate
our primary actions. We may consult potential users to determine which interfaces are the easiest to understand in an
attempt to make the interface as self-evident as possible. This stage will result in a prototype for the 'Quick
Expense' screen, one for the 'Expenses Viewer' screen, and a notion for how to tie the two screens together.

At this point we may begin to consider our implementation, and will take time to examine possible data models and
system infrastructure. We will ensure that our models are flexible enough to adapt to the inevitable changes that will
occur during the concrete implementation stage.

Once we have an understanding of how our application will be linked together we may begin the implementation stage. We
will start by implementing an MVP (minimum viable product) which has the bare minimum functionality in each section so
that we can get a feel for how the application operates. We will then perform progressive enhancement on each component
until we are satisfied.

We will require a REST-ful API for any data that will be persisted. We will also need to implement the views and
behaviour for the 'Quick Expense' and 'Expenses Viewer' screens. Our user experience is a top priority so 
significant time will be spent improving the look and feel of the experience through styling and animations.

\section{Evaluation}

To evaluate how our web application performs we will compare it to a commercial application. Users
will be asked to perform a series of tasks, using the NASA TLX questionnaire to evaluate their experience
on both applications. The TLX questionnaire will be accompanied by a qualitative questionnaire 
which will focus on the ease of use of each application and whether they would like to keep using the
application that we have built.
% \bibliographystyle{acm-sigchi} \bibliography{proposal}

\end{document}
