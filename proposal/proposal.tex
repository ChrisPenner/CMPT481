\documentclass{chi2011}
\usepackage{times}
% \usepackage{url}
\usepackage{graphics}
\usepackage{color}
\usepackage[pdftex]{hyperref}
\hypersetup{%
    pdftitle={CMPT 481 Project Proposal},
pdfauthor={Birkenstocks \& Socks: Peggy Anderson, Chris Penner, Jonathan Baxter},
pdfkeywords={your keywords},
bookmarksnumbered,
pdfstartview={FitH},
colorlinks,
citecolor=black,
filecolor=black,
linkcolor=black,
urlcolor=black,
breaklinks=true,
}
\newcommand{\comment}[1]{}
\definecolor{Orange}{rgb}{1,0.5,0}
\newcommand{\todo}[1]{\textsf{\textbf{\textcolor{Orange}{[[#1]]}}}}

\pagenumbering{arabic}  % Arabic page numbers for submission.  Remove this line to eliminate page numbers for the camera ready copy

\begin{document}
% To make various LaTeX processors do the right thing with page size.
\special{papersize=8.5in,11in}
\setlength{\paperheight}{11in}
\setlength{\paperwidth}{8.5in}
\setlength{\pdfpageheight}{\paperheight}
\setlength{\pdfpagewidth}{\paperwidth}

% Use this command to override the default ACM copyright statement
% (e.g. for preprints). Remove for camera ready copy.
% \toappear{Submitted for review to CHI 2011.}

\title{CMPT481 Project Proposal}
\numberofauthors{3}
\author{
\alignauthor Peggy Anderson\\
    \email{peggy.anderson@usask.ca}
    \alignauthor Chris Penner\\
    \email{clp848@mail.usask.ca}
    \alignauthor Jonathan Baxter\\
    \email{jab231@mail.usask.ca}
}


\maketitle

\begin{abstract}
    Polygons and stuff
\end{abstract}

\category{H.5.2}{Information Interfaces and Presentation}{Miscellaneous}[Optional sub-category]

\terms{
  See list of the limited ACM 16 terms in the instructions, see http://www.sheridanprinting.com/sigchi/generalterms.htm.
}

\section{Problem & Motivation}

In 1980, the ratio for debt-to-income in Canada is 66%; that ratio passed the 150% figure in 2011. At a young age 
Canadians are taught how to count and spend money but we are not all taught how to budget effectively. Whether you're 
a business owner, student, or bringing home the bacon for your family; it's important to budget. Budgeting allows a 
person to determine if they will have enough money for what they need to do and what they would like to do. The reason
people typically tend to fall into debt is due to the fact that they're spending more than they expect to spend. If a 
person was to set aside a predetermined amount of money they'd spend a week/month and they were able to visually see what 
they're spending versus what they're saving if they don't spend the total amount then that person could potentially 
be more cautious of what they're spending recreationally.

Where people spend money every day it can be easy to forget to add smaller expenses or a large quantity of separate 
expenses. Reason being due to depending on their method of choice for budgeting, the input of their expenses may be 
difficult to execute in a timely fashion. With current applications that are purposed towards tracking 
weekly/monthly allocated spending funds on recreational purposes (non reoccuring bills/expenses,) they are not not 
very easy or engaging to use. Engaging in regards to the visualization used is not easy to distinctly see where the 
money is being spent, whether it be food/clothes/music/games/etc. Easy to use in regards to adding expenses, or navigating 
the application where there may be "too much going on". The portability of the application can also be an issue whether the 
usage is the same when using it on a desktop versus mobile device, between different mobile devices, and between different 
internet browsers. 



\begin{enumerate}
    \item Current Budgeting apps are not user friendly
    \item It's tough visualize how much we're spending on what
    \item It's tough to input expenses/costs, clunky interfaces and cost entry
    \item Difficulties discourage people from actually tracking their expenses
\end{enumerate}

(Include crappy interface photos here)

\begin{enumerate}
\item Want to display data clearly at a glance in a way that encourages action
\item Allow people to understand where they're using their money
\item Encourage saving money
\item Improve transparency in your budget
\item Reduce friction in inputting expenses so people actually do it.
\item Mobile friendly solution
\item Budgeting apps are only useful if people log ALL expenses, so it needs to be effortless.
\end{enumerate}

\section{Solution}

\begin{enumerate}
    \item Mobile first web-app
    \item Allows inputting expenses in a matter of seconds
    \item Allows at a glance view of spending habits
    \item 'Quick Expense' screen
\end{enumerate}

\subsection{Steps to Solution}

\begin{enumerate}
    \item Spec out data models
    \item Create restful API for models
    \item Create the Quick Expense Screen
    \item Create the View Expenses screen
    \item Wire up backend functionality and data-processing
    \item Make it pretty
    \item UI polish (e.g. animations, etc.)
\end{enumerate}

\section{Evaluation}

\begin{enumerate}
    \item Time a user inputting expenses on our app vs. competitors (ours should be faster)
    \item Time a user as they evaluate their spending on our app vs competitors
    \item Qualitative comparisons of ease of use vs competitors
    \item Qualitative representation of how enjoyable navigating and using the app is.
\end{enumerate}

\bibliographystyle{acm-sigchi}
\bibliography{proposal}

\end{document}
