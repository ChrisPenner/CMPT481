\documentclass{chi2011}
\usepackage[pdftex]{hyperref}
\hypersetup{%
pdftitle={Your Title},
pdfauthor={Your Authors},
pdfkeywords={your keywords},
bookmarksnumbered,
pdfstartview={FitH},
colorlinks,
citecolor=black,
filecolor=black,
linkcolor=black,
urlcolor=black,
breaklinks=true,
}

\pagenumbering{arabic}  % Arabic page numbers for submission.  Remove this line to eliminate page numbers for the camera ready copy

\begin{document}
% To make various LaTeX processors do the right thing with page size.
\special{papersize=8.5in,11in}
\setlength{\paperheight}{11in}
\setlength{\paperwidth}{8.5in}
\setlength{\pdfpageheight}{\paperheight}
\setlength{\pdfpagewidth}{\paperwidth}

% Use this command to override the default ACM copyright statement
% (e.g. for preprints). Remove for camera ready copy.
% \toappear{Submitted for review to CHI 2011.}

\title{CMPT481 Project Report}
\numberofauthors{3}
\author{
\alignauthor Peggy Anderson\\
    \email{peggy.anderson@usask.ca}
    \alignauthor Chris Penner\\
    \email{clp848@mail.usask.ca}
    \alignauthor Jonathan Baxter\\
    \email{jab231@mail.usask.ca}
}

\maketitle

\section{Problem and Motivation}
\section{Related Literature and Background}
\section{Description of The System}

    \subsection{Initial Prototype}

    The intention was to build an application that was as easy to use as
    possible, users should be able to jump right into the tasks without prior
    training and be able to complete them confidently and successfully. The
    easiest way to accomplish this was to minify the scope of the application
    to its simplest possible subset. Minimizing the number of actions available
    simplifies the interface and allows the user to navigate confidently.

    In the initial "Wizard of Oz" style paper prototype the interface was pared
    down to only 3 screens, one for each of the tasks users are allowed to
    perform:

    \begin{itemize}
        \item Add an expense
        \item View expenses
        \item Customize budget allowances
    \end{itemize}

    As with any design process it was quickly discovered during trials that
    it was unclear how to perform certain actions; for instance there was no
    functionality for cancelling or removing an expense. Each hiccup was noted
    and changes were prioritized for the next prototyping stage.

    \subsection{Adjustments}
    
    The next prototype was constructed as a fully interactive and functional
    web application using javascript, html and css. This allowed us to
    implement the functionality which was missing from our paper prototypes
    such as animations, calculations, and the use of real data.

    A critical analysys was performed on the data collected from the
    paper-prototype trials. This inspired a focus on displaying the information
    available as clearly as possible. Naturally the implementation met some
    obstacles and decisions were made to cut or work around some features.
    Features which were cut or altered include:

    \begin{itemize}
        \item Saving data across sessions
        \item Cancelling expenses
        \item Viewing individual expenses
        \item Cancel buttons/back buttons
        \item Adding additional expense categories
    \end{itemize}

    We determined that many of these features were "nice to have" however were
    not crucial for the testing stage of this level of prototype. These
    concerns are addressed again in the retrospective portion of the report.

    Overall the choice of web technologies suited the implementation well;
    particularly the ability to use the same prototype on mobile and desktop.
    A familiarity with React and Redux allowed quick bug-free edits when
    concerns regarding functionality or layout were raised. The use of Redux
    specifically simplified the problem of syncing data across the application
    trivial. These choices were made within a forward thinking context, if the
    application were to move forward to large scale deployment the code may
    progress towards that goal.

    One of the greatest difficulties faced was that of creating a responsive
    layout that allowed code re-use across Mobile and Desktop devices. The
    preference was for the interfaces to be similar enough to allow habits
    formed on one device to 

    \subsection{Retrospective}

\section{Evaluation and User Reports}
\section{Conclusion}

\bibliographystyle{acm-sigchi}
\bibliography{sample}

\end{document}
