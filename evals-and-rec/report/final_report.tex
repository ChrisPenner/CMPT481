\documentclass{chi2011}
\usepackage[pdftex]{hyperref}
\usepackage{enumitem}
\usepackage{scrextend}
\hypersetup{%
pdftitle={Your Title},
pdfauthor={Your Authors},
pdfkeywords={your keywords},
bookmarksnumbered,
pdfstartview={FitH},
colorlinks,
citecolor=black,
filecolor=black,
linkcolor=black,
urlcolor=black,
breaklinks=true,
}

\pagenumbering{arabic}  % Arabic page numbers for submission.  Remove this line to eliminate page numbers for the camera ready copy

\begin{document}
% To make various LaTeX processors do the right thing with page size.
\special{papersize=8.5in,11in}
\setlength{\paperheight}{11in}
\setlength{\paperwidth}{8.5in}
\setlength{\pdfpageheight}{\paperheight}
\setlength{\pdfpagewidth}{\paperwidth}

% Use this command to override the default ACM copyright statement
% (e.g. for preprints). Remove for camera ready copy.
% \toappear{Submitted for review to CHI 2011.}

\title{CMPT481 Project Report}
\numberofauthors{3}
\author{
\alignauthor Peggy Anderson\\
    \email{peggy.anderson@usask.ca}
    \alignauthor Chris Penner\\
    \email{clp848@mail.usask.ca}
    \alignauthor Jonathan Baxter\\
    \email{jab231@mail.usask.ca}
}

\maketitle

\section{Problem and Motivation}

In 1980, the ratio for debt-to-income in Canada is 66%; that ratio passed the 150% figure in 2011.
1 At a young age Canadians are taught how to count and spend money but we are not all taught how to
budget effectively. Whether you are a businessowner, student, or bringing home the bacon for your
family; it is important to budget. Budgeting allows a person to determine if they will have enough
money for what they need to do and what they would like to do. The reason people might fall in to
debt is the fact that they are spending more than they expected. If a person was to set aside a
predetermined amount of money that they wish to spend each week or month and could better visualize
their spending habits in relation to their saving. This will allow them to better prioritize their
budget and find areas to increase savings.

It can be easy to forget to add expenses as they occur throughout the day. Reasons for this may
include that the method of inputting their expenses is too time-consuming or difficult to use; or
that their budgeting application is not available on their mobile device. Many of the current
applications that are purposed towards tracking weekly or monthly recreational budgets (non
re-occurring bills and expenses) are not easy to use or engaging for the user. Most visualization
techniques present in other applications focus solely on the amount of money spent in each
category, however this overlooks the primary purpose of budgeting and does not deliver the
information of how the amount spent relates to the users savings goals. Many applications sacrifice
ease-of-use by adding clutter for additional (often unused) features. When navigating many of these
applications it can seem that there is ”too much going on”. Accessibility of the application can
also be an issue, disparity in interface between the Desktop and mobile versions can be a confusing
experience. Some applications are not available on all devices.

\section{Related Literature and Background}
\section{Description of The System}

    \subsection{Initial Prototype}

    The intention was to build an application that was as easy to use as
    possible, users should be able to jump right into the tasks without prior
    training and be able to complete them confidently and successfully. The
    easiest way to accomplish this was to minify the scope of the application
    to its simplest possible subset. Minimizing the number of actions available
    simplifies the interface and allows the user to navigate confidently.

    In the initial "Wizard of Oz" style paper prototype the interface was pared
    down to only 3 screens, one for each of the tasks users are allowed to
    perform:

    \begin{itemize}[noitemsep]
        \item Add an expense
        \item View expenses
        \item Customize budget allowances
    \end{itemize}

    As with any design process it was quickly discovered during trials that
    it was unclear how to perform certain actions; for instance there was no
    functionality for cancelling or removing an expense. Each hiccup was noted
    and changes were prioritized for the next prototyping stage.

    \subsection{Adjustments}
    
    The next prototype was constructed as a fully interactive and functional
    web application using Javascript, html and CSS. This allowed us to
    implement the functionality which was missing from our paper prototypes
    such as animations, calculations, and the use of real data.

    A critical analysis was performed on the data collected from the
    paper-prototype trials. This inspired a focus on displaying the information
    available as clearly as possible. Naturally the implementation met some
    obstacles and decisions were made to cut or work around some features.
    Features which were cut or altered include:

    \begin{itemize}[noitemsep]
        \item Saving data across sessions
        \item Cancelling expenses
        \item Viewing individual expenses
        \item Cancel buttons/back buttons
        \item Adding additional expense categories
    \end{itemize}

    Many features were determined to be "nice to have" but not crucial for the
    testing stage of this level of prototype. These concerns are addressed
    again in the retrospective portion of the report.

    Overall the choice of web technologies suited the implementation well;
    particularly the ability to use the same prototype on mobile and desktop.
    A familiarity with React and Redux allowed quick bug-free edits when
    concerns regarding functionality or layout were raised. The use of Redux
    specifically simplified the problem of syncing data across the application
    trivial. These choices were made within a forward thinking context, if the
    application were to move forward to large scale deployment the code may
    progress towards that goal.

    One of the greatest difficulties faced was that of creating a responsive
    layout that allowed code re-use across Mobile and Desktop devices. The
    preference was for the interfaces to be similar enough to allow habits
    formed on one device to transfer to the other. To this end CSS media-query
    rules were used extensively alongside the responsive CSS framework
    Bulma.io. Google Chrome's developer tools were instrumental in testing
    the interface on multiple mobile devices.

    Another challenge was the limited screen real-estate on mobile devices.
    Navigation items needed to be accessible without getting in the way.
    Additional complexity was added to the application to hide and show certain
    screens since there is not enough room on a mobile device for all screens
    to be showing at the same time.

    \subsection{Retrospective}

	As expected, user testing of the interactive prototype was very revealing of
    yet more flaws, but also of successes in our design. 

    The application failed to account for the change in idioms between mobile
    and desktop, i.e. users tried to edit categories or add expenses by touching
    interface elements. Most users mentioned that they would like more
    functionality, many of the requested features were originally part our the
    design but were left out due to time constraints, cluttering of the
    interface, or implementation difficulties.

    It seemed that the most difficulty was encountered in navigation between
    features and the discoverability of functionality. This suggests that this
    is an area of potentially large benefit with lower amounts of work.

    Some requested features were deliberately left out. For instance one user
    mentioned that they prefer that their budgeting app links to their bank
    account and tracks their expenses automatically. Not only was this out of
    scope for the time-frame of the project, but it is also contrary to the
    design principles of being as simple as possible. Several other features
    would be rejected for the purposes of keeping the interface clean and
    simple.

\section{Evaluation with User Reports}

	\subsection{Goals, Approach and Rational for the Evaluation}

		\subsubsection{Goals}

		The primary goal of our project is for users to use our application to continuously track their
		expenses. It is our belief that if the application is made easy to use through an intuitive
		interface and navigation then the user will also be able to input expenses quickly and efficiently.
		If the user can quickly tell us how much they have spent in a specific category after expenses have
		been entered then it confirms that the application is presenting a clear visualization of the data.
		Unfortunately the continued use of the application can not be tested within the scope of this class
		since that would require multiple interviews with the test users to evaluate their usage and a
		high-fidelity prototype.

		\subsubsection{Approach and Rational for the Evaluation}

		We will use interviews to evaluate our goals. Seeking to determine what they require to make their
		budgeting experience with our app the most enjoyable.


	\subsection{Actual Participant Pool and Other Execution Details}

	We had used a mixture of family/friends for our test cases. 

	\begin{labeling}{User 4}
		\item [User 1] An individual who has limited experience with computers.  
		\item [User 2] An individual who has a degree in computer science and may offer some valuable 
		criticisms. 
		\item [User 3] Older demographic who may have budgeted using other methods in the past.
		\item [User 4] A business student who has studied budgeting.
	\end{labeling}


	\subsection{Divergence from Milestone III Evaluation Plan}

	Put stuff here.


	\subsection{Results}

	Also put stuff here.


	\subsection{Conclusion}

	And more stuff here.

\section{Final Recommendations}

\section{Appendices}

\bibliographystyle{acm-sigchi}
\bibliography{sample}

\end{document}
