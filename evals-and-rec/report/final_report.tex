\documentclass{chi2011}
\usepackage[pdftex]{hyperref}
\usepackage{enumitem}
\usepackage{scrextend}
\hypersetup{%
pdftitle={Your Title},
pdfauthor={Your Authors},
pdfkeywords={your keywords},
bookmarksnumbered,
pdfstartview={FitH},
colorlinks,
citecolor=black,
filecolor=black,
linkcolor=black,
urlcolor=black,
breaklinks=true,
}

\pagenumbering{arabic}  % Arabic page numbers for submission.  Remove this line to eliminate page numbers for the camera ready copy

\begin{document}
% To make various LaTeX processors do the right thing with page size.
\special{papersize=8.5in,11in}
\setlength{\paperheight}{11in}
\setlength{\paperwidth}{8.5in}
\setlength{\pdfpageheight}{\paperheight}
\setlength{\pdfpagewidth}{\paperwidth}

% Use this command to override the default ACM copyright statement
% (e.g. for preprints). Remove for camera ready copy.
% \toappear{Submitted for review to CHI 2011.}

\title{CMPT481 Project Report}
\numberofauthors{3}
\author{
\alignauthor Peggy Anderson\\
    \email{peggy.anderson@usask.ca}
    \alignauthor Chris Penner\\
    \email{clp848@mail.usask.ca}
    \alignauthor Jonathan Baxter\\
    \email{jab231@mail.usask.ca}
}

\maketitle

\section{Problem and Motivation}

In 1980, the ratio for debt-to-income in Canada is 66%; that ratio passed the 150% figure in 2011.
1 At a young age Canadians are taught how to count and spend money but we are not all taught how to
budget effectively. Whether you are a businessowner, student, or bringing home the bacon for your
family; it is important to budget. Budgeting allows a person to determine if they will have enough
money for what they need to do and what they would like to do. The reason people might fall in to
debt is the fact that they are spending more than they expected. If a person was to set aside a
predetermined amount of money that they wish to spend each week or month and could better visualize
their spending habits in relation to their saving. This will allow them to better prioritize their
budget and find areas to increase savings.

It can be easy to forget to add expenses as they occur throughout the day. Reasons for this may
include that the method of inputting their expenses is too time-consuming or difficult to use; or
that their budgeting application is not available on their mobile device. Many of the current
applications that are purposed towards tracking weekly or monthly recreational budgets (non
re-occurring bills and expenses) are not easy to use or engaging for the user. Most visualization
techniques present in other applications focus solely on the amount of money spent in each
category, however this overlooks the primary purpose of budgeting and does not deliver the
information of how the amount spent relates to the users savings goals. Many applications sacrifice
ease-of-use by adding clutter for additional (often unused) features. When navigating many of these
applications it can seem that there is ”too much going on”. Accessibility of the application can
also be an issue, disparity in interface between the Desktop and mobile versions can be a confusing
experience. Some applications are not available on all devices.

\section{Related Literature and Background}
\section{Description of The System}

    \subsection{Initial Prototype}

    The intention was to build an application that was as easy to use as
    possible, users should be able to jump right into the tasks without prior
    training and be able to complete them confidently and successfully. The
    easiest way to accomplish this was to minify the scope of the application
    to its simplest possible subset. Minimizing the number of actions available
    simplifies the interface and allows the user to navigate confidently.

    In the initial "Wizard of Oz" style paper prototype the interface was pared
    down to only 3 screens, one for each of the tasks users are allowed to
    perform:

    \begin{itemize}[noitemsep]
        \item Add an expense
        \item View expenses
        \item Customize budget allowances
    \end{itemize}

    As with any design process it was quickly discovered during trials that
    it was unclear how to perform certain actions; for instance there was no
    functionality for cancelling or removing an expense. Each hiccup was noted
    and changes were prioritized for the next prototyping stage.

    \subsection{Adjustments}
    
    The next prototype was constructed as a fully interactive and functional
    web application using Javascript, html and CSS. This allowed us to
    implement the functionality which was missing from our paper prototypes
    such as animations, calculations, and the use of real data.

    A critical analysis was performed on the data collected from the
    paper-prototype trials. This inspired a focus on displaying the information
    available as clearly as possible. Naturally the implementation met some
    obstacles and decisions were made to cut or work around some features.
    Features which were cut or altered include:

    \begin{itemize}[noitemsep]
        \item Saving data across sessions
        \item Cancelling expenses
        \item Viewing individual expenses
        \item Cancel buttons/back buttons
        \item Adding additional expense categories
    \end{itemize}

    Many features were determined to be "nice to have" but not crucial for the
    testing stage of this level of prototype. These concerns are addressed
    again in the retrospective portion of the report.

    Overall the choice of web technologies suited the implementation well;
    particularly the ability to use the same prototype on mobile and desktop.
    A familiarity with React and Redux allowed quick bug-free edits when
    concerns regarding functionality or layout were raised. The use of Redux
    specifically simplified the problem of syncing data across the application
    trivial. These choices were made within a forward thinking context, if the
    application were to move forward to large scale deployment the code may
    progress towards that goal.

    One of the greatest difficulties faced was that of creating a responsive
    layout that allowed code re-use across Mobile and Desktop devices. The
    preference was for the interfaces to be similar enough to allow habits
    formed on one device to transfer to the other. To this end CSS media-query
    rules were used extensively alongside the responsive CSS framework
    Bulma.io. Google Chrome's developer tools were instrumental in testing
    the interface on multiple mobile devices.

    Another challenge was the limited screen real-estate on mobile devices.
    Navigation items needed to be accessible without getting in the way.
    Additional complexity was added to the application to hide and show certain
    screens since there is not enough room on a mobile device for all screens
    to be showing at the same time.

    \subsection{Retrospective}

	As expected, user testing of the interactive prototype was very revealing of
    yet more flaws, but also of successes in our design. 

    The application failed to account for the change in idioms between mobile
    and desktop, i.e. users tried to edit categories or add expenses by touching
    interface elements. Most users mentioned that they would like more
    functionality, many of the requested features were originally part our the
    design but were left out due to time constraints, cluttering of the
    interface, or implementation difficulties.

    It seemed that the most difficulty was encountered in navigation between
    features and the discoverability of functionality. This suggests that this
    is an area of potentially large benefit with lower amounts of work.

    Some requested features were deliberately left out. For instance one user
    mentioned that they prefer that their budgeting app links to their bank
    account and tracks their expenses automatically. Not only was this out of
    scope for the time-frame of the project, but it is also contrary to the
    design principles of being as simple as possible. Several other features
    would be rejected for the purposes of keeping the interface clean and
    simple.

\section{Evaluation with User Reports}

	\subsection{Goals, Approach and Rational for the Evaluation}

		\subsubsection{Goals}

		The primary goal of our project is for users to use our application to continuously track their
		expenses. It is our belief that if the application is made easy to use through an intuitive
		interface and navigation then the user will also be able to input expenses quickly and efficiently.
		If the user can quickly tell us how much they have spent in a specific category after expenses have
		been entered then it confirms that the application is presenting a clear visualization of the data.
		Unfortunately the continued use of the application can not be tested within the scope of this class
		since that would require multiple interviews with the test users to evaluate their usage and a
		high-fidelity prototype.

		\subsubsection{Approach and Rational for the Evaluation}

		We will use interviews to evaluate our goals. Seeking to determine what they require to make their
		budgeting experience with our app the most enjoyable.


	\subsection{Actual Participant Pool and Other Execution Details}

	We had used a mixture of family/friends for our test cases. 

	\begin{labeling}{User 4}
		\item [User 1] An individual who has limited experience with computers.  
		\item [User 2] An individual who has a degree in computer science and may offer some valuable 
		criticisms. 
		\item [User 3] Older demographic who may have budgeted using other methods in the past.
		\item [User 4] A business student who has studied budgeting.
	\end{labeling}


	\subsection{Divergence from Milestone III Evaluation Plan}

	Upon implementing our Medium Fidelity-Prototype our task list had diverged from what we initially
	said our application would be able to do. Instead of only three tasks that would be able to be 
	performed, the application can now perform five different tasks:
	
	\begin{itemize}[noitemsep]
		\item Add an expense (unchanged)
		\item View expenses over a weeks/months/years time (unchanged)
		\item Change an overall budget (unchanged)
		\item Change a category name
		\item Set a budget limit for one or all categories
	\end{itemize}

	We had also said that our redesign would offer cancellation of actions to improve user experiense,
	however we did not include a cancellation option in our "Add Expense" page for the mobile interface.

	\subsection{Results}

	While examining the results from the interviews it was clear that there were both positive and 
	negative themes. The users appeared to have a preconceived notion on how our application should
	work from applications they had used in the past, even apps unrelated to budgeting. Due to this
	knowledge transfer it had caused both positive and negative feedback. 
	
	Some common themes shared amongst the users were:
	\begin{itemize}[noitemsep]
		\item Similar interfaces between mobile and desktop
		\item Easy and quick to add expenses
		\item Nice animations
	\end{itemize}
	Where once the user had learned how to use the application the fact that the desktop and mobile
	interfaces were so similar, it allowed users to switch between the two with minimal confusion. 
	The animations engaged the users when they had added expenses into the application, one user
	was surprised and said "Oh! Fancy." 
	
	\subsubsection{Negative Themes}
	\begin{itemize}[noitemsep]
		\item It should be more customizable
		\begin{itemize}[noitemsep]
			\item Pick default view
			\item Have more categories
		\end{itemize}
		\item More indepth data 
		\item Data bars are confusing
		\item Have the ability to click the label or the bar for the category to be able to edit 
		the name of the category or view more data. Cancel Buttons
	\end{itemize}
	
	The users wanted to be able to change the default view of budgeting data so they choose
	what bars of data to see whether it's weekly/monthly/yearly. If they didn't want to see bars of
	data, then they wanted the option to be able to see individual expenses made as their primary 
	view. This tied into the theme that users wanted more data available to them, they wanted the
	breakdown of each individual expense they made by what amount they spent, when they spent it 
	and where they spent it. Most of the users thought that this could currently be done by 
	clicking on the category name/bar. However, the bars had caused issues for some users as they 
	were unsure what they were growing relative to if there was no custom budget. Users wanted an 
	indicator that they were coming close to filling meeting their budget. It also proved to be 
	difficult for some users to refresh inputted data if they didn't want to save it. For example,
	on the add an expense page on the mobile interface a user cried out in frustraction as they hit
	the back button on their phone closed the application, they wanted it to go back to the 
	previous page. 

	\subsubsection{Above and Beyond}
	
	User \#2 as expected went above and beyond when supplying feedback. They had mentioned for 
	design ideas that if we were able to create a widget for mobile device and with this widget
	there could be multiple options for the user to customize. The user would be able to use the 
	widget to add an expense, to see total amount spent from the budget, see the total remaining 
	amount from the budget, or a combination of any of the above. 

	\subsection{Conclusion}

	And more stuff here.

\section{Final Recommendations}

	\begin{itemize}[noitemsep]
		\item Did our design change as a result of user
		\begin{itemize}[noitemsep]
			\item Yes:
			\begin{itemize}[noitemsep]
				\item Shifting layout of interface
				\item Mobile: change main page to be add-expense screen
				\item Move and add label for Settings button.
				\item Label any 'icons' with text
				\item Users want more customizability
				\item Users wanted the features we hadn't implemented
				\item Track individual expenses
				\item Change orientation of graphs
				\item Add cancel buttons
				\item Make it more intuitive for a touch interface (users touched labels)
			\end{itemize}
		\end{itemize}

		\item Stop now, go back to where you were 12 weeks ago, and decide whether anything has changed in your perspective. 
		\begin{itemize}[noitemsep]
			\item People think they want more functionality than we think they need.
		\end{itemize}

		\item What have you learned about the user-centered design process, over the course of the entire project? 
		\begin{itemize}[noitemsep]
			\item Should have run more trials at lower fidelity
			\item Users are dumber in some ways and smarter in others
			\item It's useful
			\item It's easier to get negative feedback rather than positive.
		\end{itemize}

		\item Did it work for you? 
		\begin{itemize}[noitemsep]
			\item Yes, we learned and were able to use user's feedback for the better
		\end{itemize}

		\item Did the methods you chose for your evaluation and prototyping get at what you were looking for? 
		\begin{itemize}[noitemsep]
			\item Yes, the interview was a good process, the medium fidelity worked well.
			\item The Low fidelity could have been done better/replaced with something else
		\end{itemize}

		\item In hindsight, would a different approach have been better? 
		\begin{itemize}[noitemsep]
			\item For low fidelity it would have been useful to do a more interactive and comprehensive prototype like axure or 
				  even more advanced paper prototypes with colours and shapes.
		\end{itemize}

		\item What were the most, and least, valuable among the activities we’ve asked you to try out, either generally or 
			  specifically for your project?
		\begin{itemize}[noitemsep]
			\item Most valuable was conducting guided interviews, these prompted lots of useful feedback from users.
			\item Low fidelity prototyping was useful in general, we re-architected our design and approach several times as a result
			\item 10x10 design was useful
			\item Time could have been shifted to allow more time for higher-fidelity testing and user engagement.
		\end{itemize}

		\item Having gone through this course, how might you approach your next interface
			  design project (whether for fun/personal or work, large or small) differently?
		\begin{itemize}[noitemsep]
			\item  Spend more time on design and planning before jumping in.
			\item Get more people to test it out rather than just myself.
			\item Interview testers and users with probing questions rather than just asking "what do you think"
		\end{itemize}
	\end{itemize}


\section{Appendices}

	\subsection{A1 - Evaluation Instrument}
	
	\subsubsection{Tasks we got the users to perform}
	\begin{itemize}[noitemsep]
		\item Add different expenses to two or three different categories
		\item After adding expenses, (go to screen) for data visualization
		\item Change the name of a category
		\item Change the overall budget limit (it will originally be set to 200\$)
		\item Change one category limit
		\item Add more expenses to that category and get it to go over the budget limit
	\end{itemize}
		

	\subsubsection{Pre-Demos}
	\begin{itemize}[noitemsep]
		\item Do you currently or have you ever budget(ed)?
		\begin{itemize}[noitemsep]
			\item (Yes - Probing Question) What method do you use?
			\item (No - Probing Question) Is there a specific reason why?
		\end{itemize}
	\end{itemize}
	
	\subsubsection{Post Desktop Demo}
	\begin{itemize}[noitemsep]
		\item Was there anything thing that you found counter intuitive?
		\item If you could change/add things about the app, what would it be? (From desktop perspective)
	\end{itemize}

	\subsubsection{Post Mobile Demo}
	\begin{itemize}[noitemsep]
		\item Was there anything thing that you found counter intuitive?
		\item If you could change/add things about the app, what would it be? (From mobile perspective)
	\end{itemize}


	\subsubsection{Post-Demos}
	\begin{itemize}[noitemsep]
		\item How did you find the mobile interface versus the desktop interface?
		\begin{itemize}[noitemsep]
			\item (Probing Question) Which was easiest to use?
			\item (Probing Question) For you what made the data more visually clear for you?
		\end{itemize}	
	\item (If they had budgeted previously) If you had tried using other budgeting apps before, how did ours compare?
		\begin{itemize}[noitemsep]
			\item (Yes - Probing Question) What app allowed you fast input?
			\item (Yes - Probing Question) What app allowed more accurate input?
			\item (Yes - Probing Question) Are you more likely to continue using ours or the other one?
			\item (No - Probing Question) Why?
		\end{itemize}
	\item (If they had not budgeted before) After testing this app, would you be likely to continue using it consistently? 
	\end{itemize}
	
	
	\subsection{A2 - Raw Data}
	\subsection{User \#2}

	\subsubsection{Pre-Demos}
	\begin{itemize}[noitemsep]
		\item Do you currently or have you ever budget(ed)?
		\begin{itemize}[noitemsep]
			\item 
				Yes. Right now User \#2 doesn't budget actively but they do use Mint.com to make sure old
				accounts don't have stuff on them and go through current accounts to ensure there are no
				unknown purchases. 
				
				In the past they used spreadsheets for projecting, and sometimes User \#2 still uses this
				method for their savings and for bigger purchases. 
				
				User \#2 says that they have a facade for budgeting. "Mint.com sucks for things that run over
				multiple months."
		\end{itemize}
	\end{itemize}
	
	\subsubsection{Desktop Demo}
	\begin{itemize}[noitemsep] 
		\item Add three expenses to three different categories
		\begin{itemize}[noitemsep]
			\item Saw animations and said "Aw thats fancy"
			\item Didn't use the arrows to increment/decrement
			\item Added negative numbers
			\item Found out he could input 'e' (didn't do anything except clear any numbers previously 
			typed) When asked why the letter User \#2 answered "only because I was allowed to"
		\end{itemize}
		\item Go to monthly view
		\begin{itemize}[noitemsep]
			\item Highlighted "Weekly View"
			\item Went into Settings then hit browser back button and that caused the application to close
		\end{itemize}
		\item Change the overall budget
		\begin{itemize}[noitemsep]
			\item Tried to add -0.10, changed to "NaN"
		\end{itemize}
		\item Max a category
		\begin{itemize}[noitemsep]
			\item Entered 9.99 then entered 0.01 and it didn't max the category
		\end{itemize}
	\end{itemize}
	
	\subsubsection{Post Desktop Demo}
	\begin{itemize}[noitemsep]
		\item Was there anything thing that you found counter intuitive?
		\begin{itemize}[noitemsep]
			\item Switching between views, the arrows on side made User \#2 thing that it would go to a 
			history. "Although the forward wouldn't make sense for this."
			\item Click category label to change the name
		\end{itemize}
		\item If you could change/add things about the app, what would it be? (From desktop perspective)
		\begin{itemize}[noitemsep]
			\item Expected a drop down to toggle views
			\item Add back button to settings page (or a cancel button)
			\item Hover over category label to see a button to edit the name
			\item Want to be able to see if you're increasing or decreasing spending of money of if theres a 
			  certain time of the year you're spending more money (VERY IMPORTANT TO USER - Charting
			  over time)
			\item Can't see actual expenses of what you're spending per day on what day
			\item Only tracking expenses and not income
		\end{itemize}
	\end{itemize}
	
	
	\subsubsection{Desktop Demo}
	\begin{itemize}[noitemsep] 
		\item Add two expenses to two different categories
		\begin{itemize}[noitemsep]
			\item Used "Add Another"
		\end{itemize}
	\item Asked to go to yearly expenses view
		\begin{itemize}[noitemsep]
			\item Didn't swipe
		\end{itemize}
	\end{itemize}

	\subsubsection{Post Mobile Demo}
	\begin{itemize}[noitemsep]
		\item Was there anything thing that you found counter intuitive?
		\begin{itemize}[noitemsep]
			\item No, but there were things I'd change
		\end{itemize}
		\item If you could change/add things about the app, what would it be? (From mobile perspective)
		\begin{itemize}[noitemsep]
			\item On Add Expense Page
			\item Make it so the category budget is visible
			\item Add back or cancel button
			\item Light colours for buttons don't look like it's disabled
			\begin{itemize}[noitemsep]
				\item Make it grey, or have a notification to say why you can't add an expense
				\item Can't do anything if there is no category selected, but can do something if there
				  is no expense inputted, weird
			\end{itemize}
			\item In settings 
			\begin{itemize}[noitemsep]
				\item Add back or cancel button
			\end{itemize}
			\item On main page
			\begin{itemize}[noitemsep]
				\item Change arrows to three dots to know to swipe
			\end{itemize}
			\item 
				It would be cool to have a widget so you can add an expense really easily without having 
				to open the app. Customizable widgets too, so if you wanted a quick add expense button, 
				or if you wanted to just see where you were budget wise(total amount spent versus 
				remaining budget amount, OR only total spent, OR only remaining budget amount)
		\end{itemize}
	\end{itemize}


	\subsubsection{Post-Demos}
	\begin{itemize}[noitemsep]
		\item How did you find the mobile interface versus the desktop interface?
		\begin{itemize}[noitemsep]
			\item It was nice that they both had the same features available. Mint.com disables some
				  features for their mobile application
			\begin{itemize}[noitemsep]
				\item (Probing Question) Which was easiest to use?
				\begin{itemize}[noitemsep]
					\item Desktop was easiest to use because I can use the refresh button to get out of the
						  settings page without having to hit "Save"
				\end{itemize}
			\item (Probing Question) For you what made the data more visually clear for you?
				\begin{itemize}[noitemsep]
					\item No difference
				\end{itemize}
			\end{itemize}
		\end{itemize}	
	\item (If they had budgeted previously) If you had tried using other budgeting apps before, how did ours compare?
		\begin{itemize}[noitemsep]
			\item It's not a fair comparison because mint.com is a professional product made by an 
				  international finance company
			\item Your app is easier to use if you are not wanting to have it tied to bank accounts, and is
				  easier to add expenses.
			\item I like Mint.com because it has the ability to track not just the value of what you spent
				  but also when and what the expense was for
		\end {itemize}
	\item (Probing Question) What app allowed you faster input?
		\begin{itemize}[noitemsep]
			\item For cash expenses your app is faster, or for expenses if you can't link an account
			\item Mint.com credit card expenses are delayed, however debit transactions are faster
		\end{itemize}
	\item (Probing Question) What app allowed more accurate input?
		\begin{itemize}[noitemsep]
			\item It's the same,  both can still see wha tthe amount is before it gets used
			\item For choosing categories your app is harder, Mint doesn't require me to worry about a
				  box (for inputting the amount)
		\end{itemize}
	\item (Probing Question) How likely to continue using ours over the other one?
		\begin{itemize}[noitemsep]
			\item It has potential, but I wouldn't currently use it.

				  If I were to use it I would want to include location and date of the expense. US is 
				  okay as is, not ideal. I want to be able to go back and change past expenses. I want
				  more data out of the program. Don't want expenses to be summed up. Want monthly budget
				  or yearly budget, don't care about weekly budget. In fact, I want monthly budget to be 
				  the default view, and it not be a sum of the weekly budget. Also the ability to add 
				  income and assets, or my specific bank accounts. 
			\begin{itemize}[noitemsep]
				\item (Probing Question) With those changes would you use this app over Mint.com 
					  consistently? 
				\begin{itemize}[noitemsep]
					\item With those changes? Yes.
				\end{itemize}
			\end{itemize}
		\end{itemize}
	\end{itemize}
	
	
	
	
	\subsection{User \#4}

	\subsubsection{Pre-Demos}
	\begin{itemize}[noitemsep]
		\item Do you currently or have you ever budget(ed)?
		\begin{itemize}[noitemsep]
			\item 
				Personally? No, maybe, techincally yes. Never followed it. User \#4 created a budget for the
				sake a creating a budget. Doesn't believe budgets to be beneficial. Here we talked about
				some pros and cons.
				\begin{itemize}[noitemsep]
					\item "I'm a seasonal worker, would make more sense to budget if I had a consistent 
					      income."
					\begin{itemize}[noitemsep]
						\item "Wouldn't it then make more sense to budget? You know you have this much 
							   money for these many months, budget to make sure you end by breaking even or
							   by ending in a surplus."
					\end{itemize}
				\end{itemize}
			\item 
				User \#4 says their way of budgeting is by asking "do I have enough money for the year?" 
				Where they are "Naturally frugal" they don't make many big purchases and when they do it 
				doesn't make a big difference. 
		\end{itemize}
	\end{itemize}
	
	\subsubsection{Desktop Demo}
	\begin{itemize}[noitemsep] 
		\item Add three expenses to three different categories
		\begin{itemize}[noitemsep]
			\item User \#4 had tried using the arrows to increment the amount he was adding but quickly 
				  changed to the num pad.
			\item Asked for clarification if he was adding an expense for a day/week/month/year 
		\end{itemize}
		\item Change the name of a category
		\begin{itemize}[noitemsep]
			\item Clicked on the category name
			\item Clicked on the drop down for adding expenses to a specific category
			\item Clicked on the settings gear and said "ahh"
		\end{itemize}
	\end{itemize}
	
	\subsubsection{Post Desktop Demo}
	\begin{itemize}[noitemsep]
		\item Was there anything thing that you found counter intuitive?
		\begin{itemize}[noitemsep]
			\item "Add an expense" text makes him think that he can click on it to add an expense, didn't
					immediately realize it was a title. 
			\item Not immediately drawn to the settings gear, clicked on the category title, then the 
				  category bar, then the category drop down (that you want to add an expense to) to try to
				  change the name of the category first.
			\item Scrolling arrows to increment added expense amount is weird, goes up by one.
		\end{itemize}
		\item If you could change/add things about the app, what would it be? (From desktop perspective)
		\begin{itemize}[noitemsep]
			\item Have the arrows to increment added expense increment amount by 10
			\item Have category names/bars clickable to change category name/add expense/see total expenses
			\item See the arrows to change between weekly/monthly/yearly be more prominent. Didn't notice 
					that until the interviewer pointed them out. 
			\item Have it be more customizable
			\begin{itemize}[noitemsep]
				\item Wants to be able to have more categories (more than 5)
			\end{itemize}
		\end{itemize}
	\end{itemize}
	
	\subsubsection{Desktop Demo}
	\begin{itemize}[noitemsep] 
		\item Add two expenses to two different categories
		\begin{itemize}[noitemsep]
			\item User \#4 missed the "Add Another" button
			\begin{itemize}[noitemsep]
				\item After he added the second expense the interviewer asked him to add another expense 
					  but this time click that button instead of "Done"
			\end{itemize}
		\end{itemize}
	\item Asked to go to yearly expenses view
		\begin{itemize}[noitemsep]
			\item Swiped immediately
		\end{itemize}
	\end{itemize}

	\subsubsection{Post Mobile Demo}
	\begin{itemize}[noitemsep]
		\item Was there anything thing that you found counter intuitive?
		\begin{itemize}[noitemsep]
			\item "Add Another" button - wasn't sure it would save first entry
			\item Pressing back button on mobile device ruined data
			\item Clicking save "what if I don't want to save it?"
			\item swiping was intuitive - didn't notice arrows though
		\end{itemize}
		\item If you could change/add things about the app, what would it be? (From mobile perspective)
		\begin{itemize}[noitemsep]
			\item In settings 
			\begin{itemize}[noitemsep]
				\item Change it so that you can see what you're typing in for a category limit
				\item Have more categories!
			\end{itemize}
			\item On main page
			\begin{itemize}[noitemsep]
				\item Click the category label/bar to add expense
				\item Don't like that you can't see budget numbers on screen 
				\item Add indicator that you're getting close to the budget limit, have a bar, half of what
				  you've spent, half what you have left in budget to spend
				\item See what you're spending day/month
			\end{itemize}
		\end{itemize}
	\end{itemize}

	\subsubsection{Post-Demos}
	\begin{itemize}[noitemsep]
		\item How did you find the mobile interface versus the desktop interface?
		\begin{itemize}[noitemsep]
			\item Gear was much better on mobile, had to figure it out still though
			\item Pretty much the same
			\item Desktop has stuff you can do from the main page, mobile has another screen (quick and easy to
				  use)
			\begin{itemize}[noitemsep]
				\item (Probing Question) Which was easiest to use?
				\begin{itemize}[noitemsep]
					\item Both were easy
				\end{itemize}
			\item (Probing Question) For you what made the data more visually clear for you?
				\begin{itemize}[noitemsep]
					\item Desktop because on mobile the screen is crammed. Have mobile main screen be add expense and data be a separate page showing all 
						  weekly/monthly/yearly on one page.
				\end{itemize}
			\end{itemize}
		\end{itemize}

		\item (If they had not budgeted before) After testing this app, would you be likely to continue using it consistently? 
		\begin{itemize}[noitemsep]
			\item No. It doesn't do much. Want to view list of individual expenses.
			\item More data, what days you spent, how much you spent, how it changes week to week.
		\end{itemize}
	\end{itemize}
	
	
	
	
	

\bibliographystyle{acm-sigchi}
\bibliography{sample}

\end{document}
